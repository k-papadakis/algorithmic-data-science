\documentclass{article}

\usepackage[utf8]{inputenc}
\usepackage{amsmath, amsfonts, amssymb}
\usepackage{amsthm}
% \usepackage{bbm}
% \usepackage{alphabeta}
% \usepackage{graphicx}
% \graphicspath{ {../output/} }
% \usepackage{hyperref}


% To make custom snippets, Ctrl+shift+P
% --> Preferences: Configure User Snippets -> latex (or latex.json)

\DeclareMathOperator{\E}{\textrm{E}}

% \newcommand{\inner}[2]{\left\langle #1 \mathrel{,} #2 \right\rangle}
% \newcommand{\norm}[1]{\left\| #1 \right\|}
% \newcommand{\T}[1]{{#1}^{\top}} %alternatively, use {#1}^{\intercal} or {#1}'
% \newcommand{\ve}[1]{\boldsymbol{#1}}


\title{Algorithmic Data Science - Exercise Series 2}
\author{
    Konstantinos Papadakis\\
    Data Science and Machine Learning 03400149\\
    konstantinospapadakis@mail.ntua.gr
}
\date{\today}


\begin{document}

\maketitle
\newpage

%%%%%%%%%%%%%%%%%%%%%%%%%
%%%%%%%%%%%%%%%%%%%%%%%%%
% EXERCISE 1
%%%%%%%%%%%%%%%%%%%%%%%%%
%%%%%%%%%%%%%%%%%%%%%%%%%
\section*{Exercise 1}

%%%%%%%%%%%%%%%%%%%%%%%%%
% 1(a)
%%%%%%%%%%%%%%%%%%%%%%%%%
\subsubsection*{(a)}
We have that 
\begin{align*}
    &h_{a,b}(x) = h_{a,b}(y)\\
    \iff& ax + b \equiv ay + b \mod{m}\\
    \iff& a(x-y) \equiv 0 \mod{m}
\end{align*}
which in the case of \(x=m, \  y = 0\) is true \(\forall a, b\)
therefore
\[P(h_{a,b}(x) = h_{a,b}(y)) = 1 > \frac{1}{m}\]
meaning that the family is not universal.

%%%%%%%%%%%%%%%%%%%%%%%%%
% 1(b)
%%%%%%%%%%%%%%%%%%%%%%%%%
\subsection*{(b)}

This exercise is \emph{Theorem 11.5} in \cite{clrs}.

Let \(x, y \in \mathbb{Z}_p: x \neq y\).

Define
\begin{align*}
    u &:= ax + b \mod{p}\\
    v &:= ay + b \mod{p}
\end{align*}

Note that \(u \neq v\)
since \(u - v \equiv a (x - y) \not\equiv 0 \mod{p}\)
because \(a \neq 0\) and \(x \not\equiv y \mod{p}\),
the later holding because by hypothesis \(x \neq y \ \textrm{and}\ x,y < p\)
Therefore, there are no collisions when we apply \(x \mapsto ax + b \mod{p}\).

We proceed to show that
\((a, b) \mapsto (ax + b \mod{p},\ ax + b \mod{p})\) is a bijection between
the pairs \((a, b) \in \mathbb{Z}_p^* \times \mathbb{Z}_p\)
and the pairs \((u, v) \in \mathbb{Z}_p \times \mathbb{Z}_p: u \neq v\).

We can solve for \(a, b\) and get a unique solution
\begin{align*}
    a &= \frac{u-v}{x-y} \mod{p}\\
    b &= r - ak \mod{p}
\end{align*}
Where \(\frac{1}{t}\) is the inverse of \(t\) in \(\mathbb{Z}_p\)

Therefore the mapping is one to one.
Since we also have that both the domain and the co-domain have
\(p(p-1)\) elements, the mapping is a bijection.
Thus, if \((a, b)\) is uniformly distributed, so is \((u, v)\).

Therefore, the probability that \(x, y \in \mathbb{Z}_p: x \neq y\) collide
is equal to the probability that \(u \equiv v \mod{m}\) collide
when \((u, v) \in \mathbb{Z}_p \times \mathbb{Z}_p: u \neq v\)
are chosen uniformly randomly.
We proceed to calculate that probability.

Given \(u\), of the \(p-1\) possible remaining values for \(v\) we have that
at most \(\lceil{\frac{p}{m}}\rceil - 1 \leq \frac{p-1}{m}\)
can collide with \(u\). 

Therefore the probability of collision is \(\leq \frac{1}{m}\),
meaning that the hash function family is universal.

%%%%%%%%%%%%%%%%%%%%%%%%%
% 1(c)
%%%%%%%%%%%%%%%%%%%%%%%%%
\subsubsection*{(c)}\label{1c}

The proof in \ref{1c} is still valid,
since \(x \in U \implies x < p\) is still valid.
Therefore the hash function family remains universal.


%%%%%%%%%%%%%%%%%%%%%%%%%
%%%%%%%%%%%%%%%%%%%%%%%%%
% EXERCISE 2
%%%%%%%%%%%%%%%%%%%%%%%%%
%%%%%%%%%%%%%%%%%%%%%%%%%
\section*{Exercise 2}

%%%%%%%%%%%%%%%%%%%%%%%%%
% 2(a)
%%%%%%%%%%%%%%%%%%%%%%%%%
\subsubsection*{(a)}

From what I understand,
the expected number of probes for a successful search is not equal to 
the expected number of probes for an insertion.
What \cite{clrs} says in the proof of \emph{Corollary 11.7} is that
the expected number of probes for an \emph{Unsuccessful} search is
equal to the expected number of probes for an insertion,
and that number is bounded above by \(\frac{1}{1-a}\)
where \(a\) is the load factor.

The expected number of probes a for a successful search on the other hand
is bounded above by \(\frac{1}{a} \ln\frac{1}{1-a}\),
as shown in \emph{Theorem 11.8}.

If the expected values were equal,
wouldn't the book bound them by the same number?

%%%%%%%%%%%%%%%%%%%%%%%%%
% 2(b)
%%%%%%%%%%%%%%%%%%%%%%%%%
\subsubsection*{(b)}

First we need to show that the expected number of probes in an unsuccessful
search is bounded by \(\frac{1}{1-a}\).

Let X be a random variable describing
the number of probes in an unsuccessful search.

We have
\begin{align*}
    \E[X]
        &= \sum_{i = 0}^\infty i \Pr(X = i)\\
        &= \sum_{i = 1}^\infty \Pr(X \geq i)
\end{align*}

Now,
\begin{align*}
    \Pr(X \geq i)
        &= \frac{n}{m} \cdot \frac{n-1}{m-1} \cdots \frac{n-i+2}{m-i+2}\\
        &\leq \left( \frac{n}{m} \right) ^ {i-1}\\
        &= a^{i-1}
\end{align*}

Therefore,
\begin{equation*}
    \E[X] \leq \sum_{i = 1}^\infty a^{i-1} = \frac{1}{1-a}
\end{equation*}

Since an element is inserted only
if there is room in the table (thus \(a < 1\)),
inserting a key requires an unsuccessful search
followed by placing the key into the first empty slot found.
Thus, the expected number of probes for an insertion
is at most \(\frac{1}{1-a}\).

We now proceed to prove the result about the successful search.

A search for a key reproduces the same probe sequence as when the
element with key was inserted.
By the above result, if the element was the (\(i+1\))st element inserted,
then the expected number of probes made in a search for it
is at most \(\frac{1}{1 - 1/m} = \frac{m}{m-i}\).

Therefore the expected number of probes for a successful search is
\begin{align*}
    \frac{1}{n} \sum_{i=0}^{n-1} \frac{m}{m-i}
        &= \frac{1}{a} \sum_{i=0}^{n-1} \frac{1}{m-i}\\
        &= \frac{1}{a} \sum_{k=m-n+1}^m \frac{1}{k}\\
        &\leq \frac{1}{a} \int_{m-n}^m \frac{1}{x} dx\\
        &= \frac{1}{a} \ln\frac{m}{m-n}\\
        &= \frac{1}{a} \ln\frac{1}{1-a}
\end{align*}

\bibliographystyle{IEEEtran}
\bibliography{references}
\end{document}